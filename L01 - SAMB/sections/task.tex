%!TEX root=../document.tex

\section{Einführung}
(Die Einführung wurde geschrieben von Kevin Waldock)

Dieses Protokoll befasst sich mit der Installation und Konfiguration von LDAP auf Ubuntu. 

\subsection{Ziele}
Nach dem Aufsetzen von OpenLDAP soll es möglich sein sich mithilfe von LDAP zu authentifizieren.

\subsection{Voraussetzungen}
Für dieses Aufgabe werden folgende Software benötigt:

\begin{itemize}
	\item VMware Workstation Pro 12 oder höher
	\item Ubuntu 14.04 LTS
\end{itemize}

Das Betriebssystem muss in der VMware Workstation bereits installiert sein.

\subsection{Aufgabenstellung}

In Gruppen von 2-3 Personen (die einzelnen Schritte müssen dokumentiert und getestet sein):

\begin{itemize}
	\item Installieren Sie zwei OpenLDAP Server
	\item Konfigurieren Sie eine Master-Slave Replikation zwischen den beiden Servern
	\item Erstellen Sie Accounts im LDAP für mindestens 2 Benutzer
	\item Erstellen Sie im LDAP eine Benutzergruppe mit den zuvor angelegten Benutzern
	\item Konfigurieren Sie ein Client-System für die Authentifizierung über LDAP  (entweder global oder nur für spezielle Services wie z.B. FTP oder SSH)
\end{itemize}

\clearpage
