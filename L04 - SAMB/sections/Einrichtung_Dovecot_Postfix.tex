%!TEX root=../document.tex

\section{Ergebnisse}
\label{sec:Ergebnisse}

\subsection{Installation von Dovecot und Postfix}
(Geschrieben von Kevin Waldock)

\subsubsection{Vorbereitungen}
Bevor die Installation von Dovcot und Postfix durchgeführt wird, sollte das System up-to-date sein.
Daher sollten folgende Befehle zunächst ausgeführt werden:

\verb|sudo -s|\newline
\verb|apt update && apt upgrade|

Als nächstes muss der Hostname der Maschine gesetzt werden. Dafür wird \verb|/etc/hostname| geändert mit dem folgenen Inhalt:

\verb|syt|

Parallel dazu, muss auch die hosts-Datei aktualisiert werden:
\begin{lstlisting}[escapechar=@]
127.0.1.1	syt.at syt
127.0.0.1	localhost
#127.0.1.1	mzaher

# The following lines are desirable for IPv6 capable hosts
::1 syt.at ip6-localhost 
#::1     ip6-localhost ip6-loopback
#fe00::0 ip6-localnet
#ff00::0 ip6-mcastprefix
#ff02::1 ip6-allnodes
#ff02::2 ip6-allrouters
\end{lstlisting}

Anstatt \verb|mzaher| werden die hostnames \verb|syt.at| und \verb|syt| verwendet. Nach den Änderugen ist ein neustart der Maschine erforderlich.

Beim Ausführen von \verb|hostname -f| sollte nun \verb|syt.at| ausgegeben werden.

\subsubsection{Test-SSL-Zertifikat}
Damit wir SSL verwenden können, müssten wir ein Zertifikat kaufen. Für diesen Testlauf generieren wir ein Test-SSL-Zertifikat mit dem Kommando \verb|sudo make-ssl-cert generate-default-snakeoil|. Fallst \verb|make-ssl-cert| nicht existiert, dann muss es noch installiert werden mit \newline\verb|sudo apt-get install ssl-cert|.

Dadurch werden zwei Dateien erstellt: \newline
\verb|/etc/ssl/certs/ssl-cert-snakeoil.pem| \newline
\verb|/etc/ssl/private/ssl-cert-snakeoil.key| \newline

Diese werden dann bei der weiteren Konfiguration verwendet.

\subsubsection{Installation der Packete}

% TODO: Evtl. noch weiter Konfigurationsdetails?

Für die Datenbank für Postfix wird mysql verwendet. Daher muss zunächst das MySQL-Package installiert werden:\newline
\verb|sudo apt-get install mysql-server|

Als nächstes werden die Dovecot-Packete installiert:\newline
\verb|sudo apt-get install dovecot-core dovecot-imapd dovecot-lmtpd dovecot-mysql |\newline
\verb|dovecot-sieve dovecot-managesieved dovecot-antispam|\newline

Zum Schluss wird dann noch Postfix und die MySQL-Integration installiert: \newline
\verb|sudo apt-get install postfix postfix-mysql|

Bei der Frage nach der ``Allgemeinen Art der Konfiguration'' muss ``Keine Konfiguration'' ausgewählt werden. Dies ist dann noch später wichtig.

\subsubsection{Einrichten der MySQL Datenbank für Postfix}
In Linux kann man sich mit der MySQL-Shell direkt mit seinem Root-Passwort anmelden. Dazu kann der Befehl:\newline
\verb|mysql -u root -p|\newline
verwendet werden.

Wenn angemeldet, wird zunächst eine neue Datenbank erstellt:\newline
\verb|create database vmail;|\newline
\verb|vmail| ist die Datenbank, auf die dann Postfix zugreifen kann.

Als nächstes muss ein User erstellt werden, mit welchem sich Postfix auf die Datenbank anmelden kann:\newline
\verb|grant all on vmail.* to 'vmail'@'localhost' identified by '12345qwert';|\newline
Das Passwort sollte entsprechen lang sein, um auch Sicherheit zu gewähren. In dieser Demonstration wurde nur das Passwort \verb|12345qwert| gewählt.

\paragraph{Tabellen erstellen}\mbox{}\\
Damit Postfix funktioniert, müssen die entsprechenden Tabellen erstellt werden, die dann die Konfigurations- und Benutzerdaten speichert.

Dafür muss zunächst die Datenbank ausgewählt werden mit:\newline
\verb|use vmail|\newline

\subparagraph{Domain-Tabelle}\mbox{}\\
\begin{lstlisting}[language=SQL, caption=vmail - Domain-Tabelle]
CREATE TABLE `domains` (
	`id` int unsigned NOT NULL AUTO_INCREMENT,
	`domain` varchar(255) NOT NULL,
	PRIMARY KEY (`id`),
	UNIQUE KEY (`domain`)
) CHARSET=latin1;
\end{lstlisting}
Diese Tabelle enthält alle Domains, die der Mailserver verwendet soll.

\subparagraph{Account-Tabelle}\mbox{}\\
\begin{lstlisting}[language=SQL, caption=vmail - Account-Tabelle]
CREATE TABLE `accounts` (
	`id` int unsigned NOT NULL AUTO_INCREMENT,
	`username` varchar(64) NOT NULL,
	`domain` varchar(255) NOT NULL,
	`password` varchar(255) NOT NULL,
	`quota` int unsigned DEFAULT '0',
	`enabled` boolean DEFAULT '0',
	`sendonly` boolean DEFAULT '0',
	PRIMARY KEY (id),
	UNIQUE KEY (`username`, `domain`),
	FOREIGN KEY (`domain`) REFERENCES `domains` (`domain`)
) CHARSET=latin1;
\end{lstlisting}
Diese Tabelle repräsentiert alle Mailserver-Accounts.

\subparagraph{Alias-Tabelle}\mbox{}\\
\begin{lstlisting}[language=SQL, caption=vmail - Alias-Tabelle]
CREATE TABLE `aliases` (
	`id` int unsigned NOT NULL AUTO_INCREMENT,
	`source_username` varchar(64) NOT NULL,
	`source_domain` varchar(255) NOT NULL,
	`destination_username` varchar(64) NOT NULL,
	`destination_domain` varchar(255) NOT NULL,
	`enabled` boolean DEFAULT '0',
	PRIMARY KEY (`id`),
	UNIQUE KEY (`source_username`, `source_domain`, `destination_username`, `destination_domain`),
	FOREIGN KEY (`source_domain`) REFERENCES `domains` (`domain`)
) CHARSET=latin1;
\end{lstlisting}
Diese Tabelle repräsentiert alle Aliase. Damit können Weiterleitungen erstellt werden.

\subparagraph{TLS Policy-Tabelle}\mbox{}\\
\begin{lstlisting}[language=SQL, caption=vmail - Alias-Tabelle]
CREATE TABLE `tlspolicies` (
	`id` int unsigned NOT NULL AUTO_INCREMENT,
	`domain` varchar(255) NOT NULL,
	`policy` enum('none', 'may', 'encrypt', 'dane', 'dane-only', 'fingerprint', 'verify', 'secure') NOT NULL,
	`params` varchar(255),
	PRIMARY KEY (`id`),
	UNIQUE KEY (`domain`)
) CHARSET=latin1;
\end{lstlisting}
Diese Tabelle bestimmt die Sicherheitsbeschränkungen.

\subsubsection{Einrichtung der Benutzer und Verzeichnisse für Postfix}
Postfix benötigt für den Zugriff auf MySQL eine eigenen Benutzer. Der Benutzer wird \verb|vmail| heißen, da dieser auch schon für MySQL konfiguriert wurde.

Als Home-Verzeichnis wird \verb|/var/vmail| verwendet, wo dann alle Mailserver-Daten (E-Mails, Filter, ...) gespeichert werden. Daher muss zunächst dieser Ordner erstellt werden:\newline
\verb|mkdir /var/vmail|\newline

Als nächstes muss der Systembenutzer erstellt werden:\newline
\verb|adduser --disabled-login --disabled-password --home /var/vmail vmail|\newline
Wichtig ist, dass der richtige Home-Verzeichnis angegeben wird.

Dann werden noch die übrigen Unterverzeichnisse erstellt:\newline
\verb|mkdir /var/vmail/mailboxes|\newline
\verb|mkdir -p /var/vmail/sieve/global|\newline

Zum Schluss muss nur noch die Benutzerrechte für die Ordner angepasst werden, so dass auch \verb|vmail| auf diese Ordner zurgreifen kann.\newline
\verb|chown -R vmail /var/vmail|\newline
\verb|chgrp -R vmail /var/vmail|\newline
\verb|chmod -R 770 /var/vmail|\newline


\subsubsection{Einrichtung von Dovecot}
Dovecot wird benötigt, damit Protokolle wie IMAP mit Postfix funktionieren.

Für die Einrichtung muss zunächst Dovecot gestoppt werden:\newline
\verb|systemctl stop dovecot|\newline

Prinzipielle könnte man jetzt jede einzelne Konfiguratonsdatei editieren. Da Dovecot aber nur lediglich zwei Konfigurationsdateien braucht (dovecot.conf, dovecot-sql.conf), ist es besser die Konfigurationsorder zu entleeren, und dann mit den kommenden vorgefertigten Dateien zu arbeiten.
s
Um den Konfigurationsordner zu entleeren müssen folgende Befehle ausgeführt werden:\newline
\verb|rm -r /etc/dovecot/*|\newline
\verb|cd /etc/dovecot|\newline

Jetzt können die Dateien erstellt werden.

% TODO: Kürzen/Highlighting?
\paragraph{dovecot.conf}\mbox{}\\
\begin{lstlisting}[caption=dovecot.conf]
###
### Aktivierte Protokolle
#############################

protocols = imap lmtp sieve



###
### TLS Config
#######################

ssl = required
ssl_cert = </etc/ssl/certs/ssl-cert-snakeoil.pem
ssl_key = </etc/ssl/private/ssl-cert-snakeoil.key
ssl_dh_parameters_length = 2048
ssl_protocols = !SSLv2 !SSLv3
ssl_cipher_list = EDH+CAMELLIA:EDH+aRSA:EECDH+aRSA+AESGCM:EECDH+aRSA+SHA256:EECDH:+CAMELLIA128:+AES128:+SSLv3:!aNULL:!eNULL:!LOW:!3DES:!MD5:!EXP:!PSK:!DSS:!RC4:!SEED:!IDEA:!ECDSA:kEDH:CAMELLIA128-SHA:AES128-SHA
ssl_prefer_server_ciphers = yes



###
### Dovecot services
################################

service imap-login {
    inet_listener imap {
        port = 143
    }
}


service managesieve-login {
    inet_listener sieve {
        port = 4190
    }
}


service lmtp {
    unix_listener /var/spool/postfix/private/dovecot-lmtp {
        mode = 0660
        group = postfix
        user = postfix
    }

    user = vmail
}


service auth {
    ### Auth socket fuer Postfix
    unix_listener /var/spool/postfix/private/auth {
        mode = 0660
        user = postfix
        group = postfix
    }

    ### Auth socket fuer LMTP-Dienst
    unix_listener auth-userdb {
        mode = 0660
        user = vmail
        group = vmail
    }
}


###
###  Protocol settings
#############################

protocol imap {
    mail_plugins = $mail_plugins quota imap_quota antispam
    mail_max_userip_connections = 20
    imap_idle_notify_interval = 29 mins
}

protocol lmtp {
    postmaster_address = postmaster@syt.at #geaendert
    mail_plugins = $mail_plugins sieve
}



###
### Client authentication
#############################

disable_plaintext_auth = yes
auth_mechanisms = plain login


passdb {
    driver = sql
    args = /etc/dovecot/dovecot-sql.conf
}

userdb {
    driver = sql
    args = /etc/dovecot/dovecot-sql.conf
}


###
### Mail location
#######################

mail_uid = vmail
mail_gid = vmail
mail_privileged_group = vmail


mail_home = /var/vmail/mailboxes/%d/%n
mail_location = maildir:~/mail:LAYOUT=fs



###
### Mailbox configuration
########################################

namespace inbox {
    inbox = yes

    mailbox Spam {
        auto = subscribe
        special_use = \Junk
    }

    mailbox Trash {
        auto = subscribe
        special_use = \Trash
    }

    mailbox Drafts {
        auto = subscribe
        special_use = \Drafts
    }

    mailbox Sent {
        auto = subscribe
        special_use = \Sent
    }
}



###
### Mail plugins
############################


plugin {
    sieve_before = /var/vmail/sieve/global/spam-global.sieve   
    sieve = /var/vmail/sieve/%d/%n/active-script.sieve
    sieve_dir = /var/vmail/sieve/%d/%n/scripts

    quota = maildir:User quota
    quota_exceeded_message = Benutzer %u hat das Speichervolumen ueberschritten. / User %u has exhausted allowed storage space.

    antispam_backend = pipe
    antispam_spam = Spam
    antispam_trash = Trash
    antispam_pipe_program = /var/vmail/spampipe.sh
    antispam_pipe_program_spam_arg = --spam
    antispam_pipe_program_notspam_arg = --ham
}
\end{lstlisting}

% TODO: Beschreibung?

\paragraph{dovecot-sql.conf}\mbox{}\\
\begin{lstlisting}[caption=dovecot-sql.conf]
driver=mysql
connect = "host=127.0.0.1 dbname=vmail user=vmail password=12345qwert"
default_pass_scheme = PLAIN

password_query = SELECT username AS user, domain, password FROM accounts WHERE username = '%n' AND domain = '%d' and enabled = true;

user_query = SELECT concat('*:storage=', quota, 'M') AS quota_rule FROM accounts WHERE username = '%n' AND domain = '%d' AND sendonly = false;

iterate_query = SELECT username, domain FROM accounts where sendonly = false;
\end{lstlisting}

Hier wird die Schnittstelle zwischen der MySQL-Datenbank und dovecot definiert. In Praxis sollte \verb|default_pass_scheme = PLAIN| nicht angewendet werden. Zur einfachen Einrichtung eines E-Mail-Clients bzw. zu Testzwecken wurde das Passwort-Schema auf \verb|PLAIN| gesetzt. 

\subsubsection{Einrichtung von Postfix}
Für die Einrichtung von Postfix muss zunächst sichergestellt sein, dass Postfix deaktiviert wurde:
\verb|systemctl stop postfix|\newline

Auch wenn bei der Installation ``Keine Konfiguration'' ausgewählt wurde, sind trotzdem unter \verb|/etc/postfix| noch Konfigurationsdateien zu finden. Daher müssen folgende Befehle ausgeführt werden:\newline
\verb|cd /etc/postfix|\newline
\verb|rm -r sasl|\newline
\verb|rm master.cf|\newline

\begin{lstlisting}[caption=main.cf - Postfix]
##
## Netzwerkeinstellungen
##

mynetworks = 127.0.0.0/8 [::ffff:127.0.0.0]/104 [::1]/128
inet_interfaces = all
myhostname = syt.at
mydestination = localhost

##
## Mail-Queue Einstellungen
##

maximal_queue_lifetime = 1h
bounce_queue_lifetime = 1h
maximal_backoff_time = 15m
minimal_backoff_time = 5m
queue_run_delay = 5m


##
## TLS Einstellungen
###

tls_ssl_options = NO_COMPRESSION
tls_high_cipherlist = EDH+CAMELLIA:EDH+aRSA:EECDH+aRSA+AESGCM:EECDH+aRSA+SHA256:EECDH:+CAMELLIA128:+AES128:+SSLv3:!aNULL:!eNULL:!LOW:!3DES:!MD5:!EXP:!PSK:!DSS:!RC4:!SEED:!IDEA:!ECDSA:kEDH:CAMELLIA128-SHA:AES128-SHA

### Ausgehende SMTP-Verbindungen (Postfix als Sender)

smtp_tls_security_level = dane
smtp_dns_support_level = dnssec
smtp_tls_policy_maps = mysql:/etc/postfix/sql/tls-policy.cf
smtp_tls_session_cache_database = btree:${data_directory}/smtp_scache
smtp_tls_protocols = !SSLv2, !SSLv3
smtp_tls_ciphers = high
smtp_tls_CAfile = /etc/ssl/certs/ca-certificates.crt


### Eingehende SMTP-Verbindungen

smtpd_tls_security_level = may
smtpd_tls_protocols = !SSLv2, !SSLv3
smtpd_tls_ciphers = high
smtpd_tls_dh1024_param_file = /etc/myssl/dh2048.pem
smtpd_tls_session_cache_database = btree:${data_directory}/smtpd_scache

smtpd_tls_cert_file=/etc/letsencrypt/live/mail.mysystems.tld/fullchain.pem
smtpd_tls_key_file=/etc/letsencrypt/live/mail.mysystems.tld/privkey.pem


##
## Lokale Mailzustellung an Dovecot
##

virtual_transport = lmtp:unix:private/dovecot-lmtp


##
## Milter: DKIM-Signaturen durch OpenDKIM-Milter
## und Mail-Filter mit Amavis (via amavisd-milter)
##

milter_default_action = accept
milter_protocol = 2
smtpd_milters = unix:/var/run/amavis/amavisd-milter.sock,
unix:/var/run/opendkim/opendkim.sock
non_smtpd_milters = unix:/var/run/opendkim/opendkim.sock

##
## Server Restrictions fuer Clients, Empfaenger und Relaying
## (im Bezug auf S2S-Verbindungen. Mailclient-Verbindungen werden in master.cf im Submission-Bereich konfiguriert)
##

### Bedingungen, damit Postfix als Relay arbeitet (fuer Clients)
smtpd_relay_restrictions =      reject_non_fqdn_recipient
reject_unknown_recipient_domain
permit_mynetworks
reject_unauth_destination


### Bedingungen, damit Postfix ankommende E-Mails als Empfaengerserver entgegennimmt (zusaetzlich zu relay-Bedingungen)
### check_recipient_access prueft, ob ein account sendonly ist
smtpd_recipient_restrictions = check_recipient_access mysql:/etc/postfix/sql/recipient-access.cf


### Bedingungen, die SMTP-Clients erfuellen muessen (sendende Server)
smtpd_client_restrictions =     permit_mynetworks
check_client_access hash:/etc/postfix/without_ptr
reject_unknown_client_hostname


### Wenn fremde Server eine Verbindung herstellen, muessen sie einen gueltigen Hostnamen im HELO haben.
smtpd_helo_required = yes
smtpd_helo_restrictions =   permit_mynetworks
reject_invalid_helo_hostname
reject_non_fqdn_helo_hostname
reject_unknown_helo_hostname

# Clients blockieren, wenn sie versuchen zu frueh zu senden
smtpd_data_restrictions = reject_unauth_pipelining



##
## Postscreen Filter
##

### Postscreen Whitelist / Blocklist
postscreen_access_list =        permit_mynetworks
cidr:/etc/postfix/postscreen_access
postscreen_blacklist_action = drop


# Verbindungen beenden, wenn der fremde Server es zu eilig hat
postscreen_greet_action = drop


### DNS blocklists
postscreen_dnsbl_threshold = 2
postscreen_dnsbl_sites = dnsbl.sorbs.net*1, bl.spamcop.net*1, ix.dnsbl.manitu.net*2, zen.spamhaus.org*2
postscreen_dnsbl_action = drop


##
## MySQL Abfragen
##
virtual_alias_maps = mysql:/etc/postfix/sql/aliases.cf
virtual_mailbox_maps = mysql:/etc/postfix/sql/accounts.cf
virtual_mailbox_domains = mysql:/etc/postfix/sql/domains.cf
local_recipient_maps = $virtual_mailbox_maps


##
## Sonstiges
##

### Maximale Groesse der gesamten Mailbox (soll von Dovecot festgelegt werden, 0 = unbegrenzt)
mailbox_size_limit = 0

### Maximale Groesse eingehender E-Mails in Bytes (50 MB)
message_size_limit = 52428800

### Keine System-Benachrichtigung fuer Benutzer bei neuer E-Mail
biff = no

### Nutzer muessen immer volle E-Mail Adresse angeben - nicht nur Hostname
append_dot_mydomain = no

### Trenn-Zeichen fuer "Address Tagging"
recipient_delimiter = +
\end{lstlisting}

Wichtig sind vor allem folgende Zeilen:
\verb|mynetworks = 127.0.0.0/8 [::ffff:127.0.0.0]/104 [::1]/128|\newline
\verb|inet_interfaces = all|\newline
\verb|myhostname = syt.at|\newline
\verb|mydestination = localhost|\newline

\verb|mynetworks| definiert Addressebreich(e) die von Postfix anders behandelt werden.\newline
\verb|inet_interfaces| sagt aus für welche IP-Adresse(n) Postfix die Ports öffnen soll. Bei \verb|all| werden alle IP-Adressen berücksichtigt.\newline
\verb|myhostname| ist der hostname des Mailservers.\newline
\verb|mydestination| sagt aus wo die E-Mails landen soll. Bei \verb|localhost| landen die E-Mails lokal auf den Rechner.\newline


Als nächstes muss die ``master.cf'' angelegt werden, die die Definition und Konfiguration der verschiedenen Postfix-Dienste bereitstellt. Der Inhalt der Datei lautet wie folgt:
\begin{lstlisting}[caption=master.cf - Postfix]
# ==========================================================================
# service type  private unpriv  chroot  wakeup  maxproc command + args
#               (yes)   (yes)   (no)    (never) (100)
# ==========================================================================

###
### Postscreen-Service: Prüft eingehende SMTP-Verbindungen auf Spam-Server
###
smtp      inet  n       -       n       -       1       postscreen
-o smtpd_sasl_auth_enable=no
###
### SMTP-Daemon hinter Postscreen: Schleift E-Mails zur Filterung durch Amavis
###
smtpd     pass  -       -       n       -       -       smtpd
-o smtpd_sasl_auth_enable=no
###
### dnsblog führt DNS-Abfragen für Blocklists durch
###
dnsblog   unix  -       -       n       -       0       dnsblog
###
### tlsproxy gibt Postscreen TLS support
###
tlsproxy  unix  -       -       n       -       0       tlsproxy
###
### Submission-Zugang für Clients: Für Mailclients gelten andere Regeln, als für andere Mailserver (siehe smtpd_ in main.cf)
###
submission inet n       -       n       -       -       smtpd
-o syslog_name=postfix/submission
-o smtpd_tls_security_level=encrypt
-o smtpd_sasl_auth_enable=yes
-o smtpd_sasl_type=dovecot
-o smtpd_sasl_path=private/auth
-o smtpd_sasl_security_options=noanonymous
-o smtpd_relay_restrictions=reject_non_fqdn_recipient,reject_unknown_recipient_domain,permit_mynetworks,permit_sasl_authenticated,reject
-o smtpd_sender_login_maps=mysql:/etc/postfix/sql/sender-login-maps.cf
-o smtpd_sender_restrictions=permit_mynetworks,reject_non_fqdn_sender,reject_sender_login_mismatch,permit_sasl_authenticated,reject
-o smtpd_client_restrictions=permit_mynetworks,permit_sasl_authenticated,reject
-o smtpd_helo_required=no
-o smtpd_helo_restrictions=
-o milter_macro_daemon_name=ORIGINATING
-o cleanup_service_name=submission-header-cleanup
###
### Weitere wichtige Dienste für den Serverbetrieb
###
pickup    unix  n       -       n       60      1       pickup
cleanup   unix  n       -       n       -       0       cleanup
qmgr      unix  n       -       n       300     1       qmgr
tlsmgr    unix  -       -       n       1000?   1       tlsmgr
rewrite   unix  -       -       n       -       -       trivial-rewrite
bounce    unix  -       -       n       -       0       bounce
defer     unix  -       -       n       -       0       bounce
trace     unix  -       -       n       -       0       bounce
verify    unix  -       -       n       -       1       verify
flush     unix  n       -       n       1000?   0       flush
proxymap  unix  -       -       n       -       -       proxymap
proxywrite unix -       -       n       -       1       proxymap
smtp      unix  -       -       n       -       -       smtp
relay     unix  -       -       n       -       -       smtp
showq     unix  n       -       n       -       -       showq
error     unix  -       -       n       -       -       error
retry     unix  -       -       n       -       -       error
discard   unix  -       -       n       -       -       discard
local     unix  -       n       n       -       -       local
virtual   unix  -       n       n       -       -       virtual
lmtp      unix  -       -       n       -       -       lmtp
anvil     unix  -       -       n       -       1       anvil
scache    unix  -       -       n       -       1       scache
submission-header-cleanup unix n - n    -       0       cleanup
-o header_checks=regexp:/etc/postfix/submission_header_cleanup
\end{lstlisting}




\subsubsection{Testung mit Thunderbird}


\subsubsection{Quellen}
\url{https://thomas-leister.de/mailserver-unter-ubuntu-16.04}


