%!TEX root=../document.tex

\section{Ergebnisse}
\label{sec:Ergebnisse}

\subsection{Installation von Dovecot und Postfix}
(Geschrieben von Kevin Waldock)

\subsubsection{Vorbereitungen}
Bevor die Installation von Dovcot und Postfix durchgeführt wird, sollte das System up-to-date sein.
Daher sollten folgende Befehle zunächst ausgeführt werden:

\verb|sudo -s|\newline
\verb|apt update && apt upgrade|

Als nächstes muss der Hostname der Maschine gesetzt werden. Dafür wird \verb|/etc/hostname| geändert mit dem folgenen Inhalt:

\verb|syt|

Parallel dazu, muss auch die hosts-Datei aktualisiert werden:
\begin{lstlisting}[escapechar=@]
127.0.1.1	syt.at syt
127.0.0.1	localhost
#127.0.1.1	mzaher

# The following lines are desirable for IPv6 capable hosts
::1 syt.at ip6-localhost 
#::1     ip6-localhost ip6-loopback
#fe00::0 ip6-localnet
#ff00::0 ip6-mcastprefix
#ff02::1 ip6-allnodes
#ff02::2 ip6-allrouters
\end{lstlisting}

Anstatt \verb|mzaher| werden die hostnames \verb|syt.at| und \verb|syt| verwendet. Nach den Änderugen ist ein neustart der Maschine erforderlich.

Beim Ausführen von \verb|hostname -f| sollte nun \verb|syt.at| ausgegeben werden.

\subsubsection{Test-SSL-Zertifikat}
Damit wir SSL verwenden können, müssten wir ein Zertifikat kaufen. Für diesen Testlauf generieren wir ein Test-SSL-Zertifikat mit dem Kommando \verb|sudo make-ssl-cert generate-default-snakeoil|. Fallst \verb|make-ssl-cert| nicht existiert, dann muss es noch installiert werden mit \newline\verb|sudo apt-get install ssl-cert|.

Dadurch werden zwei Dateien erstellt: \newline
\verb|/etc/ssl/certs/ssl-cert-snakeoil.pem| \newline
\verb|/etc/ssl/private/ssl-cert-snakeoil.key| \newline

Diese werden dann bei der weiteren Konfiguration verwendet.

\subsubsection{Installation der Packete}

% TODO: Evtl. noch weiter Konfigurationsdetails?

Für die Datenbank für Postfix wird mysql verwendet. Daher muss zunächst das MySQL-Package installiert werden:\newline
\verb|sudo apt-get install mysql-server|

Als nächstes werden die Dovecot-Packete installiert:\newline
\verb|sudo apt-get install dovecot-core dovecot-imapd dovecot-lmtpd dovecot-mysql |\newline
\verb|dovecot-sieve dovecot-managesieved dovecot-antispam|\newline

Zum Schluss wird dann noch Postfix und die MySQL-Integration installiert: \newline
\verb|sudo apt-get install postfix postfix-mysql|

\subsubsection{Einrichten der MySQL Datenbank für Postfix}
In Linux kann man sich mit der MySQL-Shell direkt mit seinem Root-Passwort anmelden. Dazu kann der Befehl:\newline
\verb|mysql -u root -p|\newline
verwendet werden.

Wenn angemeldet, wird zunächst eine neue Datenbank erstellt:\newline
\verb|create database vmail;|\newline
\verb|vmail| ist die Datenbank, auf die dann Postfix zugreifen kann.

Als nächstes muss ein User erstellt werden, mit welchem sich Postfix auf die Datenbank anmelden kann:\newline
\verb|grant all on vmail.* to 'vmail'@'localhost' identified by '12345qwert';|\newline
Das Passwort sollte entsprechen lang sein, um auch Sicherheit zu gewähren. In dieser Demonstration wurde nur das Passwort \verb|12345qwert| gewählt.

\paragraph{Tabellen erstellen}\mbox{}\\
Damit Postfix funktioniert, müssen die entsprechenden Tabellen erstellt werden, die dann die Konfigurations- und Benutzerdaten speichert.

Dafür muss zunächst die Datenbank ausgewählt werden mit:\newline
\verb|use vmail|\newline

\subparagraph{Domain-Tabelle}\mbox{}\\
\begin{lstlisting}[language=SQL, caption=vmail - Domain-Tabelle]
CREATE TABLE `domains` (
	`id` int unsigned NOT NULL AUTO_INCREMENT,
	`domain` varchar(255) NOT NULL,
	PRIMARY KEY (`id`),
	UNIQUE KEY (`domain`)
) CHARSET=latin1;
\end{lstlisting}
Diese Tabelle enthält alle Domains, die der Mailserver verwendet soll.

\subparagraph{Account-Tabelle}\mbox{}\\
\begin{lstlisting}[language=SQL, caption=vmail - Account-Tabelle]
CREATE TABLE `accounts` (
	`id` int unsigned NOT NULL AUTO_INCREMENT,
	`username` varchar(64) NOT NULL,
	`domain` varchar(255) NOT NULL,
	`password` varchar(255) NOT NULL,
	`quota` int unsigned DEFAULT '0',
	`enabled` boolean DEFAULT '0',
	`sendonly` boolean DEFAULT '0',
	PRIMARY KEY (id),
	UNIQUE KEY (`username`, `domain`),
	FOREIGN KEY (`domain`) REFERENCES `domains` (`domain`)
) CHARSET=latin1;
\end{lstlisting}
Diese Tabelle repräsentiert alle Mailserver-Accounts.

\subparagraph{Alias-Tabelle}\mbox{}\\
\begin{lstlisting}[language=SQL, caption=vmail - Alias-Tabelle]
CREATE TABLE `aliases` (
	`id` int unsigned NOT NULL AUTO_INCREMENT,
	`source_username` varchar(64) NOT NULL,
	`source_domain` varchar(255) NOT NULL,
	`destination_username` varchar(64) NOT NULL,
	`destination_domain` varchar(255) NOT NULL,
	`enabled` boolean DEFAULT '0',
	PRIMARY KEY (`id`),
	UNIQUE KEY (`source_username`, `source_domain`, `destination_username`, `destination_domain`),
	FOREIGN KEY (`source_domain`) REFERENCES `domains` (`domain`)
) CHARSET=latin1;
\end{lstlisting}
Diese Tabelle repräsentiert alle Aliase. Damit können Weiterleitungen erstellt werden.

\subparagraph{TLS Policy-Tabelle}\mbox{}\\
\begin{lstlisting}[language=SQL, caption=vmail - Alias-Tabelle]
CREATE TABLE `tlspolicies` (
	`id` int unsigned NOT NULL AUTO_INCREMENT,
	`domain` varchar(255) NOT NULL,
	`policy` enum('none', 'may', 'encrypt', 'dane', 'dane-only', 'fingerprint', 'verify', 'secure') NOT NULL,
	`params` varchar(255),
	PRIMARY KEY (`id`),
	UNIQUE KEY (`domain`)
) CHARSET=latin1;
\end{lstlisting}
Diese Tabelle bestimmt die Sicherheitsbeschränkungen.

\subsubsection{Einrichtung der Benutzer und Verzeichnisse für Postfix}
Postfix benötigt für den Zugriff auf MySQL eine eigenen Benutzer. Der Benutzer wird \verb|vmail| heißen, da dieser auch schon für MySQL konfiguriert wurde.

Als Home-Verzeichnis wird \verb|/var/vmail| verwendet, wo dann alle Mailserver-Daten (E-Mails, Filter, ...) gespeichert werden. Daher muss zunächst dieser Ordner erstellt werden:\newline
\verb|mkdir /var/vmail|\newline

Als nächstes muss der Systembenutzer erstellt werden:\newline
\verb|adduser --disabled-login --disabled-password --home /var/vmail vmail|\newline
Wichtig ist, dass der richtige Home-Verzeichnis angegeben wird.

Dann werden noch die übrigen Unterverzeichnisse erstellt:\newline
\verb|mkdir /var/vmail/mailboxes|\newline
\verb|mkdir -p /var/vmail/sieve/global|\newline

Zum Schluss muss nur noch die Benutzerrechte für die Ordner angepasst werden, so dass auch \verb|vmail| auf diese Ordner zurgreifen kann.\newline
\verb|chown -R vmail /var/vmail|\newline
\verb|chgrp -R vmail /var/vmail|\newline
\verb|chmod -R 770 /var/vmail|\newline










\subsubsection{Quellen}
\url{https://thomas-leister.de/mailserver-unter-ubuntu-16.04}


