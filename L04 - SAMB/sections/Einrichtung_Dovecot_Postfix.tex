%!TEX root=../document.tex

\section{Ergebnisse}
\label{sec:Ergebnisse}

\subsection{Installation von Dovecot und Postfix}
(Geschrieben von Kevin Waldock)

\subsubsection{Vorbereitungen}
Bevor die Installation von Dovcot und Postfix durchgeführt wird, sollte das System up-to-date sein.
Daher sollten folgende Befehle zunächst ausgeführt werden:

\verb|sudo -s|\newline
\verb|apt update && apt upgrade|

Als nächstes muss der Hostname der Maschine gesetzt werden. Dafür wird \verb|/etc/hostname| geändert mit dem folgenen Inhalt:

\verb|syt|

Parallel dazu, muss auch die hosts-Datei aktualisiert werden:
\begin{lstlisting}[escapechar=@]
127.0.1.1	syt.at syt
127.0.0.1	localhost
#127.0.1.1	mzaher

# The following lines are desirable for IPv6 capable hosts
::1 syt.at ip6-localhost 
#::1     ip6-localhost ip6-loopback
#fe00::0 ip6-localnet
#ff00::0 ip6-mcastprefix
#ff02::1 ip6-allnodes
#ff02::2 ip6-allrouters
\end{lstlisting}

Anstatt \verb|mzaher| werden die hostnames \verb|syt.at| und \verb|syt| verwendet. Nach den Änderugen ist ein neustart der Maschine erforderlich.

Beim Ausführen von \verb|hostname -f| sollte nun \verb|syt.at| ausgegeben werden.

\subsubsection{Test-SSL-Zertifikat}
Damit wir SSL verwenden können, müssten wir ein Zertifikat kaufen. Für diesen Testlauf generieren wir ein Test-SSL-Zertifikat mit dem Kommando \verb|sudo make-ssl-cert generate-default-snakeoil|. Fallst \verb|make-ssl-cert| nicht existiert, dann muss es noch installiert werden mit \newline\verb|sudo apt-get install ssl-cert|.

Dadurch werden zwei Dateien erstellt

